%%
%% This is file `sample-sigconf.tex',
%% generated with the docstrip utility.
%%
%% The original source files were:
%%
%% samples.dtx  (with options: `sigconf')
%% 
%% IMPORTANT NOTICE:
%% 
%% For the copyright see the source file.
%% 
%% Any modified versions of this file must be renamed
%% with new filenames distinct from sample-sigconf.tex.
%% 
%% For distribution of the original source see the terms
%% for copying and modification in the file samples.dtx.
%% 
%% This generated file may be distributed as long as the
%% original source files, as listed above, are part of the
%% same distribution. (The sources need not necessarily be
%% in the same archive or directory.)
%%
%%
%% Commands for TeXCount
%TC:macro \cite [option:text,text]
%TC:macro \citep [option:text,text]
%TC:macro \citet [option:text,text]
%TC:envir table 0 1
%TC:envir table* 0 1
%TC:envir tabular [ignore] word
%TC:envir displaymath 0 word
%TC:envir math 0 word
%TC:envir comment 0 0
%%
%%
%% The first command in your LaTeX source must be the \documentclass command.
\documentclass[sigconf]{acmart}

%%
%% \BibTeX command to typeset BibTeX logo in the docs
\AtBeginDocument{%
  \providecommand\BibTeX{{%
    Bib\TeX}}}

%% Rights management information.  This information is sent to you
%% when you complete the rights form.  These commands have SAMPLE
%% values in them; it is your responsibility as an author to replace
%% the commands and values with those provided to you when you
%% complete the rights form.
\setcopyright{acmcopyright}
\copyrightyear{2018}
\acmYear{2018}
\acmDOI{XXXXXXX.XXXXXXX}

%% These commands are for a PROCEEDINGS abstract or paper.
\acmConference[Conference acronym 'XX]{Make sure to enter the correct
  conference title from your rights confirmation emai}{June 03--05,
  2018}{Woodstock, NY}
\acmPrice{15.00}
\acmISBN{978-1-4503-XXXX-X/18/06}


%%
%% Submission ID.
%% Use this when submitting an article to a sponsored event. You'll
%% receive a unique submission ID from the organizers
%% of the event, and this ID should be used as the parameter to this command.
%%\acmSubmissionID{123-A56-BU3}

%%
%% For managing citations, it is recommended to use bibliography
%% files in BibTeX format.
%%
%% You can then either use BibTeX with the ACM-Reference-Format style,
%% or BibLaTeX with the acmnumeric or acmauthoryear sytles, that include
%% support for advanced citation of software artefact from the
%% biblatex-software package, also separately available on CTAN.
%%
%% Look at the sample-*-biblatex.tex files for templates showcasing
%% the biblatex styles.
%%

%%
%% The majority of ACM publications use numbered citations and
%% references.  The command \citestyle{authoryear} switches to the
%% "author year" style.
%%
%% If you are preparing content for an event
%% sponsored by ACM SIGGRAPH, you must use the "author year" style of
%% citations and references.
%% Uncommenting
%% the next command will enable that style.
%%\citestyle{acmauthoryear}



%%
%% end of the preamble, start of the body of the document source.
\begin{document}

%%
%% The "title" command has an optional parameter,
%% allowing the author to define a "short title" to be used in page headers.
\title{Evaluating Ad Creative and Web Context Alignment with Attention Measurement}

%%
%% The "author" command and its associated commands are used to define
%% the authors and their affiliations.
%% Of note is the shared affiliation of the first two authors, and the
%% "authornote" and "authornotemark" commands
%% used to denote shared contribution to the research.
\author{John Hawkins}
\email{john.hawkins@playgroundxyz.com}
\orcid{1234-5678-9012}
\author{Graham Burton}
\email{graham.burton@playgroundxyz.com}
\affiliation{%
  \institution{Playground XYZ}
  \streetaddress{Level 11/61 York St}
  \city{Sydney}
  \state{NSW}
  \country{Australian}
  \postcode{2000}
}

\renewcommand{\shortauthors}{Hawkins \& Burton}

\begin{abstract}
Contextual targeting is a common strategy that places marketing messages
in media locations that are aligned with a target audience.
The challenge of contextual targeting is knowing the ideal schema and
the set of categories that provide the right audience.
Refinement of the contextual targeting process has been
limited by the use of metrics that are either rapid but
unreliable (click through rates), or reliable
but slow, expensive and inaccessible in real-time (conversions or brand awareness).

In this study we evaluate the use of attention measurement as a technique for
understanding the alignment between advertising creatives and media contexts.
We perform the analysis using
a large scale dataset of impressions categorised using the iAB V2.0 taxonomy. We
evaluate the alignment by looking for consistency across time in the attention
a given creative attracts over the levels of the categorisation hierarchy.

Our research is the largest known study of the relationship between advertising creative,
media context and consumer attention. We provide strong empirical evidence that consumer
attention is related to an alignment between the creative execution and topics of media
context. Furthermore, we show that this relationship transcends any specific publication
by providing reliable indications of attention on new website domains.

\end{abstract}

%%
%% The code below is generated by the tool at http://dl.acm.org/ccs.cfm.
%% Please copy and paste the code instead of the example below.
%%
\begin{CCSXML}
<ccs2012>
 <concept>
  <concept_id>10010520.10010553.10010562</concept_id>
  <concept_desc>Computer systems organization~Embedded systems</concept_desc>
  <concept_significance>500</concept_significance>
 </concept>
 <concept>
  <concept_id>10010520.10010575.10010755</concept_id>
  <concept_desc>Computer systems organization~Redundancy</concept_desc>
  <concept_significance>300</concept_significance>
 </concept>
 <concept>
  <concept_id>10010520.10010553.10010554</concept_id>
  <concept_desc>Computer systems organization~Robotics</concept_desc>
  <concept_significance>100</concept_significance>
 </concept>
 <concept>
  <concept_id>10003033.10003083.10003095</concept_id>
  <concept_desc>Networks~Network reliability</concept_desc>
  <concept_significance>100</concept_significance>
 </concept>
</ccs2012>
\end{CCSXML}

\ccsdesc[500]{Computer systems organization~Embedded systems}
\ccsdesc[300]{Computer systems organization~Redundancy}
\ccsdesc{Computer systems organization~Robotics}
\ccsdesc[100]{Networks~Network reliability}

\keywords{Contextual Targeting, Attention Measurement, Digital Advertising, Digital Media}

\maketitle

\section{Introduction}
Contextual targeting allows advertisers to reach people with known interests,
rendered explicit
by the choice to consume media with content about a specific topic.
These contexts are thereby used to infer potential purchase intentions,
due to the association
between interests and products. Context is increasingly seen as a critical
factor in creating and defining brand engagement \cite{Wang2006}.

Context is also often used as an indirect method of reaching people that belong to certain
audiences or groups on the basis that people's interests and demographics are aligned \cite{str2016}.
A strategy that may be more effective given that targeting on the basis of gender alone
may not drive better results for many campaigns \cite{Jansen2013}.
The underlying assumptions about the relationships between demographics,
content and intentions
are supported by investigations into the social media behaviour of different demographic
groups \cite{Wang2016}.
Multiple studies have looked at the effect of media context on advertising outcomes.
A recent meta-analysis found significant effects for multiple dimensions of
media context on advertising recall \cite{Kwon2019}.
However, studies that look at the congruence
between the topic context of a page and the content of an ad have found some
counterintuitive results. For
example, brand recall can be associated with ads appearing within irrelevant contexts,
potentially due to an ability to stand out from a suite of relevant ads \cite{Kononova2020}.

The difficulty of implementing effective contextual targeting comes down three key issues.
You will need to determine an ideal contextual hierarcy or categorisation scheme that can be
applied uniformly across all media. You need to identify which components of that categorisation
scheme are appropriate for your brand or specific campaign. Finally, you need to balance the
contextual targeting against other goals like reach and frequency.
Advertisers generally must expand targeting to across high-level categories in order to
fulfill requirements as highly specific contexts are both expensive and rare.

In this work we experiment with the use of gaze duration as a measurement of
visual attention to determine the relative performance of categories within
the iAB V2.0 Contextual Hierarchy \cite{iabv2}.
We undertake this to understand the uniformity of the right context for a given brand, and
the reliability of performance across the hierarchy and across time. The goal is to determine
strategies by which contextual targeting can be refined for advertising outcomes.

Eye tracking studies are widely used as a method of measuring overt attention
to visual stimuli and have been applied to study the extent to which people
look at, and remember, advertising \cite{Hervet2011}. The technology has also
allowed the study of many factors that contribute to effective marketing,
including the impact of images of faces \cite{Djamasbi2010},
the use of animation \cite{Hamborg2012,Lee2012}, the relationship with
social media posts \cite{Barreto2013}, and emotional responses to visual stimuli
in different physical spaces \cite{Kim2021}.

Multiple investigations have identifed that attention time on advertising
provides a real-time metric that is correlated with important downstream
impacts like brand recall \cite{Lee2012}, consideration and purchase intent (\citeauthor{dentsu,PlaygroundXYZ}). Importantly, even though low attention can drive brand outcomes,
multiple studies have demonstrated that greater attention through direct
fixation has a greater impact on marketing outcomes, including
consideration, choice \cite{Santoso2021} and short term advertising
strength \cite{NelsonField}. The pervasive evidence of the role of visual attention
in the success of advertising has led to increasing focus on Attention Based Marketing
research \cite{Orquin2020}.

Contextual relevance is a method of aligning the message in advertising with the editorial
content that it is presented with \cite{Wang2011}. This is not simply a matter of
being \emph{about} the same topic, as consumers's have been shown to
to pay more attention to advertising that is relevant to their current task
(i.e. what they are looking for, or why they are reading) than simply advertising that
is contextually aligned \cite{Resnick2016}. This is coherent with observations
in the psychological literature that overt visual attention is influenced by motivations
and personality traits of individuals \cite{Kaspar2011}. Thus, the role of contextual
targeting may be to bring \emph{top-down} attention selection to advertising that would
otherwise garner attention in a \emph{bottom-up} manner \cite{Greenberg2013}.

We draw on two observations to motivate the current work. Firstly, advertisers are aware
that alignment between a campaign creative and media context can drive outcomes.
Secondly, they require methods for detecting when that alignment is functioning and
when it is not. The thesis of this research is that large scale real-time
measurement of ad attention can determine if an ad is aligned with a specific context,
and thereby provide a tool for fine tuning contextual targeting
strategies for advertiser outcomes. Conversely, it opens the door to using contextual
targeting as a method for advertisers to combat diminishing attention on
advertising \cite{Teixeira2014,Orquin2020}.

\section{Methodology}

In this research we investigate the efficacy of contextual targeting strategies
for obtaining improvements in the mean attention time paid to digital advertising.
The essential research question is whether the attention as ad attracts
within a given media context provides information about its future performance.
The null model assumes that context is not relevant and any contextual
variation should be subject to a regression to the mean.
In other words, the control hypothesis is that the mean attention paid to
a specific creative should be a better predictor of
future attention, than the attention that was paid within a specific context.

For the purposes of obtaining large scale measurements of attention time on
display advertising we utilise a machine learning system trained on data
collected from eye-tracking studies.
The attention model provides an impression level estimate of attention time,
that aggregates up into low error measurements of mean attention
time over groups of inventory.

\subsection{Attention Measurement}

We developed an attention measurement system that predicts attention time from user
behavioural signals. The training data is collected from eye tracking panels in
which subjects read media in self-directed sessions while eye tracking data is collected
using a gaze fixation model trained on facial images \cite{Krafka2016}.

As the eye tracking data is collected, we track a large number of other signals
about the user experience including environmental signals like the page structure,
and behavioural data like scroll events and the position of the advertising
in the viewport.

\begin{table}
  \caption{Attention Measurement Model - Aggregate Error}
  \label{tab:atm}
\begin{tabular}{|l|r|r|}
\toprule
Metric                  &Banner       &MREC       \\
\midrule
Aggregate MAE           &10.5 ms      &12.5 ms    \\
95\% Upper Bound        &19.0 ms      &20.6 ms    \\
95\% Lower Bound        &-22.8 ms     &-31.1 ms   \\
  \bottomrule
\end{tabular}
\end{table}

The signals collected along with the eye tracking data allow us to build and deploy an
attention measurment model that can predict the attention time paid to
specific ad units within a page. The attention model can
be applied to all digital inventory that accepts the javascript tag.

Our model evaluation process focuses on the accuracy of measuring mean
attention time for a specific ad format, over varying samples of inventory.
To generate the performance statistics we run bootstrap sampling of the test data
to look at the expected error when predicting the mean attention time over
impressions for a specific ad format. Each metric is calculated by taking 50 impression
samples 2000 times, and using the resulting distribution of mean error.
The mean attention time measurement performance is summarised in Table
\ref{tab:atm}.
We see that the aggregate mean absolute error (MAE) for both formats is very low
at $10.5 ms$ and $12.5 ms$ for banners and MRECs respectively. Furthermore, the
$95\%$ confidence intervals for these values is tightly constrained to be well under
$50 ms$ either side of the true value.
We apply this attention measurement model across large scale digitial inventory to
collect a data set of attention time paid to a range of advertisments over time.
We use these attention measured impressions to investigate the
impact of contextual categories of media on the attention paid to advertising.

\subsection{Data}
The data used in this study was collected from a wide range of advertisers running
broadly targeted campaigns between February and March 2022. It includes brands in
the automotive, retail, travel and health services verticals.
The data is broken into two one week time periods, the raw data was filtered
such that we had impressions for every creative in both
periods of time. This was a requirement of our approach so that we can
study the continuity of the relationships between creative, context and attention
across time.

In the analysis that follows we apply different levels of aggregation to the
log level impression data, so that we calculate the mean attention time for
a given creative in the specific context. Within each experiment we will
filter the aggregated data to ensure that every sample point is derived from
a least 50 impressions. This is done to ensure robust estimates of mean attention within
a given category.
Summary statistics for the dataset are shown in Table \ref{tab:dataset}. Note, that
due to the restriction outlined above, not every sample point will be in all experiemnts.

\begin{table}
\caption{Dataset of Attention Measured Impressions}
\label{tab:dataset}
\begin{tabular}{|l|r|r|}
\toprule
Metric          &Feb 18 - 24    &Feb 25 - March 03          \\
\midrule
Impressions     &16,895,661     &10,488,816     \\
URLS            &26,006         &27,855         \\
Domains         &2,263          &2,585          \\
Creatives       &683            &683            \\
Advertisers     &59             &59             \\
\bottomrule
\end{tabular}
\end{table}

\subsection{Contextual Categorisation}

In order to derive contextual categories for each record in our dataset,
we applied the Verity API (\citeauthor{verity}) to the URLs. We converted the results of
the API response into a variety of features that describe the contextual
categories derived from the iAB V2.0 hierarchy \cite{iabv2}.

We identified the iAB category for which Verity had returned the
highest score and converted that category into a set of labels for
classification across the four Tiers of iAB V2.0. In instances where the
classified category is from a higher position in the hierarchy, then the
lower level tiers are given the deepest category. For example, if we have
a URL that is classified as iABv2 category "Business and Finance"
(a Tier 1 category) then Tiers 2, 3 and 4 will also contain the same value.

Alternatively, if we have another URL that is classified as "Marketing and Advertising"
(a Tier3 category below "Business and Finance"), then the hierarchical labels
would be assigned as follows:

\begin{itemize}
    \item {\texttt{Tier 1}}: ["Business and Finance"]
    \item {\texttt{Tier 2}}: ["Business"]
    \item {\texttt{Tier 3}}: ["Marketing and Advertising"]
    \item {\texttt{Tier 4}}: ["Marketing and Advertising"]
\end{itemize}


The idea of this scheme is that every page will be assigned a categorisation across
all four levels in the hierarchy, but sometimes the category at a given Tier
will not be any more specific than the parent category. This approach ensures
that comparisons can always be made, and when we look at a deeper level in the hierarchy
we will have at least as many distinct categories as the parent.

\subsection{Metrics}

The focus of our analysis is to determine whether the performance of each creative,
in a set of contexts, remains consistent across the two time periods.
Within each time period we aggregate
the measured attention time over the impression level data to obtain multiple metrics
for the reliability of attention time within a given context.


The attention metrics are calculated for a set of impressions $I_p$ in a time period $p$.
We first define a baseline for a creative $c$ in period $p$ as:
${}^c\mathcal{A}_p$. This is defined in equation \ref{campaign_baseline}.

\begin{equation}
\label{campaign_baseline}
{}^c\mathcal{A}_p =  \frac{ \sum_{i \in I_p} \mathbf{1}^c(i) A_i }{ \sum_{i \in I_p} \mathbf{1}^c(i) }
\end{equation}

Where the function $A(i)$ returns the attention measured on impression $i$ and
$\mathbf{1}^c(i)$ is an indicator function that
tests whether impression $i$ belongs to creative $c$, as shown in equation \ref{indicator}.

\begin{equation}
\label{indicator}
  \mathbf{1}^{c}(i):=
  \begin{cases}1~&{\text{ if }}~i\in c~,\\0~&{\text{ if }}~i\notin c~.\end{cases}
\end{equation}

Similarly, we denote the baseline attention achieved by ads on pages classified
as belonging to category $t$ in period $p$ as: ${}_t\mathcal{A}_p$. Defined in
equation \ref{topic_baseline}. Note, we use the notation $t$ for categories as they
are equivalent to topics, and this distinguishes them from the creatives.


\begin{equation}
\label{topic_baseline}
{}_t\mathcal{A}_p =  \frac{ \sum_{i \in I_p} \mathbf{1}_t(i) A(i) }{ \sum_{i \in I_p} \mathbf{1}_t(i) }
\end{equation}

Where $\mathbf{1}_t(i)$ is an indicator function that
tests whether impression $i$ was shown on a page classified as category $t$,
as show in in equation \ref{indicator2}.

\begin{equation}
\label{indicator2}
  \mathbf{1}_{t}(i):=
  \begin{cases}1~&{ i\text{ categorised as }}~t~,\\0~&{\text{ otherwise }}.\end{cases}
\end{equation}

We denote the mean attention performance for a creative $c$, on media of category $t$
within period $p$ as ${}^c_t\mathcal{A}_p$.
This is defined as shown in Equation \ref{mean_attention}.

\begin{equation}
\label{mean_attention}
{}^c_t\mathcal{A}_p =  \frac{ \sum_{i \in I_p} \mathbf{1}^c_t(i) A(i) }{ \sum_{i \in I_p} \mathbf{1}^c_t(i) }
\end{equation}

Where we simply apply a indicator function $\mathbf{1}^c_t(i)$ that is a composite
of Equations \ref{indicator} and \ref{indicator2} to filter for inventory that
belongs both to creative $c$ and category $t$. These equations should illustrate that we
make our evaluations by looking at the expected attention time across difference slices
of the inventory.

\subsection{Analysis}

Our analysis proceeds by ranking all topics that each creative $c$ was exposed
to according to the value of ${}^c_t\mathcal{A}_1$.
We then look at the same creatives in period 2 and
evaluate ${}^c_t\mathcal{A}_2$ for the top three ranking categories from the first period,
as well as the bottom ranking category.
This approach necessitates that each creative be exposed to a minimum
of these same 4 categories in both periods.
In the results that will follow we report the number
of records in each trial, reflecting the number of aggregated records
that fulfills these requirements.

Once we have the datasets in each period that meet these requirements we use multiple metrics
to evaluate the impact of the contextual categories. Our primary metric is the percentage
of instances in which the category specific performance ${}^c_t\mathcal{A}_2$ is above the
creative's baseline expected performance ${}^c\mathcal{A}_1$.
This is shown as \textbf{\% Over} (the baseline) in the tables of results.

In addition we look at the mean difference between ${}^c_t\mathcal{A}_2$ and ${}^c\mathcal{A}_1$
over all creatives in the experiment. This is denoted \boldmath$\mu\delta$ (or mean delta) in
the tables and defined in Equation \ref{delta}

\begin{equation}
\label{delta}
\mu\delta = \frac{ \sum_{c \in \mathcal{C}} {}^c_t\mathcal{A}_2 - {}^c\mathcal{A}_1 }{ \mathcal{C} }
\end{equation}

The mean delta illustrates an expectation of the difference between attention time performance
and the creative baseline, using a specific contextual targeting strategy. In addition, we show
that difference expressed as a mean percentage (against the baseline) in the column
$\mathbf{\mu\%\delta}$. This can be interpreted as the expected percentage increase/decrease
relative to the creative's baseline attention, when using that contextual targeting strategy.


\section{Results}

We conduct two variations of the experiment, each one looking at different
criteria for evaluating performance on new inventory.
These are referred to as the
\emph{No URL Repeats} and the \emph{No Domain Repeats} experiments.

\subsection{No URL Repeats}

In Table \ref{tab:nourlrepeats} we see the results of looking at performance on novel
URLs for a creative, where the iAB V2.0 contextual categories of those URLs were present
in either the top three or last ranking position in the initial period.

\begin{table}
\caption{Attention Performance by Category Rank on Novel URLs}
\label{tab:nourlrepeats}
\begin{tabular}{|l|r|l|r|r|r|}
\toprule
Level   &Count    &Rank &\% Over  &$\mu\delta$  &$\mu\%\delta$        \\
\midrule
Tier1   &206      &1st  &100.0\%    &4.5         &93.2\%     \\
        &         &2nd  &96.6\%     &2.6         &49.8\%     \\
        &         &3rd  &80.6\%     &1.4         &28.7\%     \\
        &         &Last &0.0\%      &-3.7        &-53.3\%     \\
Tier2   &210      &1st  &99.5\%     &6.3         &132.0\%     \\
        &         &2nd  &95.7\%     &3.7         &68.6\%     \\
        &         &3rd  &80.0\%     &2.4         &41.2\%     \\
        &         &Last &0.0\%      &-4.1        &-59.2\%     \\
Tier3   &210      &1st  &99.5\%     &6.4         &135.7\%     \\
        &         &2nd  &95.7\%     &3.8         &69.5\%     \\
        &         &3rd  &79.5\%     &2.4         &41.8\%     \\
        &         &Last &0.0\%      &-4.1        &-59.2\%     \\
Tier4   &210      &1st  &99.5\%     &6.4         &135.7\%     \\
        &         &2nd  &95.7\%     &3.8         &69.5\%     \\
        &         &3rd  &79.5\%     &2.4         &41.8\%     \\
        &         &Last &0.0\%      &-4.1        &-59.2\%     \\
\bottomrule
\end{tabular}
\end{table}

We see that, in general, targeting pages using categories deeper into the hierarchy
improves the expected increase in attention through contextual targeting.
However, the effect appears to plateau at Tier3, likely due to the small proportion of inventory
that is uniquely classified at a Tier 4 level.

\begin{figure*}
\includegraphics[scale=0.6]{../ranking_analysis/results/iABv2/no_url_repeat.png}
\caption{Expected Lift in Attention Time on Inventory without Repeated URLs.}
\label{fig:no_url}
\end{figure*}

We show the change in attention lift across the dimensions of iAB Tiers and category ranking
in Figure \ref{fig:no_url}. This figure demonstrates the increased performance with the specificity
of the iAB Tiers and the reduction in performance with the movement down category ranks.

\subsection{No Domain Repeats}
We repeat the analysis restricting ourselves to impressions in the second period
that involve URLS for which the domain was not present in the URLs for the first period.
The results are shown in Table \ref{tab:nodomainrepeats}.
We note that the number of records has dropped by
more than half in this experiment, due to the difficulty of finding domains that were
not present in the first period.

\begin{table}
\caption{Attention Performance by Category Rank on Novel Domains}
\label{tab:nodomainrepeats}
\begin{tabular}{|l|r|l|r|r|r|}
\toprule
Level   &Count  &Rank &\% Over  &$\mu\delta$  &$\mu\%\delta$        \\
\midrule
Tier1   &86       &1st  &98.8\%     &5.2         &144.3\%     \\
        &         &2nd  &83.7\%     &2.3         &58.8\%     \\
        &         &3rd  &60.5\%     &0.4         &26.9\%     \\
        &         &Last &0.0\%      &-4.1        &-64.5\%     \\
Tier2   &83       &1st  &96.4\%     &6.4         &159.1\%     \\
        &         &2nd  &84.3\%     &2.6         &71.0\%     \\
        &         &3rd  &59.0\%     &0.7         &35.8\%     \\
        &         &Last &0.0\%      &-4.5        &-70.0\%     \\
Tier3   &82       &1st  &96.3\%     &6.5         &161.3\%     \\
        &         &2nd  &84.2\%     &2.7         &71.2\%     \\
        &         &3rd  &56.1\%     &0.6         &34.7\%     \\
        &         &Last &0.0\%      &-4.6        &-69.9\%     \\
Tier4   &82       &1st  &96.3\%     &6.5         &161.3\%     \\
        &         &2nd  &84.2\%     &2.7         &71.6\%     \\
        &         &3rd  &56.1\%     &0.6         &34.7\%     \\
        &         &Last &0.0\%      &-4.6        &-69.9\%     \\
\bottomrule
\end{tabular}
\end{table}


We show the change in attention lift across the dimensions of iAB Tiers and category ranking
in Figure \ref{fig:no_dom}. This figure demonstrates that in the no domain repeat experiment
we get a very similar overall pattern to the previous experiment. The primary difference
being a higher expected lift at the rank one position, and a faster drop off in expected
lift as we move down the ranks.

\begin{figure*}
\includegraphics[scale=0.6]{../ranking_analysis/results/iABv2/no_dom_repeat.png}
\caption{Expected Lift in Attention Time on Inventory without Repeated Domains.}
\label{fig:no_dom}
\end{figure*}

The restriction to unique domains in the second period has lifted the expected improvement
of targeting the top performing category (Tier 3) from 135% to 161%. However, this improvement
only occurs with the category in ranking position one. For the other ranking positions
the results are mixed, even though the percentage amount is sometimes higher, the
absolute amount can be lower, due to the fact that the baseline is calculated on
the impressions that qualify for inclusion.

These result seems counterintuitive at first glance. Our current hypothesis about why
this is the case is that it relates to the specificity of the content on a given domain.
The \emph{no domain repeats} study is biased towards data on less common domains.
These may have a tendency to be more subject specific.
These ideas need to be tested in follow up research.

Additionally, we note that the percentage of instances where the category targeting
delivers an improvement over the baseline drops when we look at unique domains.
This drop is approximately 3% for rank 1 categories, but by the time we arrive at
the 3rd ranked category it has dropped from 80% to 56% at Tier 4, meaning that
targeting new domains with the 3rd ranked category offers a marginally better than
random chance of delivering improved attention.

\subsection{Significance Tests}

We estimated p-values for the significance of the proportion of URLs within a given category
that outperform the baseline. We did this using a binomial distribution, under the null model
assumption that the relationship between the attention on a URL and its contextual category
is random. Hence, the p-value can be estimated as the probability of obtaining the observed
number of successful improvements over the baseline as samples from a binomial with a probability
of success equal to $0.5$.

The computed p-values were extremely small, starting at $10^{-62}$ for first rank of Tier 1 in
the \emph{No URL Repeats} experiment, $10^{-24}$ for the \emph{No Domain Repeats} experiment.
The p-values increase gradually as we move down the ranking of the categories or at deeper
levels in the iAB hierarchy, but they always remain below a significance level
of 0.05.

A more stringent analysis would apply a Bonferroni correction to account for the multiple
hypothesis tests in these tables of data. Even with this correction, the vast majority
of results
are many orders of magnitude below the significance threshold. However, under this criteria
we would reject the third ranking category within the \emph{No Domain Repeats} experiments.
However, as the observed data is overwhelmingly statistically significant, and these values
are consistent with the hypothesis being true, but suffering from a diminishing effect size,
and reduced sample sizes, we included the values for the third ranking categories in the
table.

\section{Conclusion}
The pattern we observe is consistent between the two experiments. Regardless of
how novel the inventory is, there is a tendency for media sharing the top contextual
categories from the
initial period to consistently outperform the baseline attention.
This pattern is strongest with the number one category, and then recedes
as we progress down the ranking of categories from the first period.

Similarly, the bottom performing category from the first period consistently under performs
the baseline in the subsequent period, regardless of whether you are looking at new URLs
on existing domains, or entirely new domains.
These results strongly suggest that the topics discussed on pages in which an ad appears
interact with ad creatives in a way that enables prediction of ad attention.

As we move deeper into the hierarchy from Tier 1 down to Tier 4 we see that the average size
of the improvement increases, but this effect appears to plateau at Tier 3. We note that this
improvement would often be countered by reduced inventory volumes, suggesting that application
of these results will require weighting the gains against potential reach.

When we look at inventory on novel domains we see that there is a further small drop in the
proportion of instances where the category continues to deliver improved results. This is
countered by an improvement in the gain for the top performing category, suggesting that a
reasonable and risk efficient strategy for finding new inventory is to target multiple top
ranking categories on known websites, but just the top ranking category on novel websites.

Overall, these results provide strong evidence that attention measurement provides a signal
as to whether a given contextual targeting strategy is aligned with an advertising creative.
Performance, in terms of attention garnered, in an initial period is predictive of
attention in a future period even when the specific media changes.
The reliability of attention on new URLs or domains within the same contextual category
suggests that the iAB hierarchy provides insight into factors that align advertising
creatives with consumer psychology beyond the specifics of a particular media platform.

\section{Acknowledgments}
This research was financially supported by Playground XYZ and Gumgum.
The content benefited from commentary and observations made by Rob Hall.

\bibliographystyle{ACM-Reference-Format}
\bibliography{../bibliography/refs}

\end{document}
\endinput
%%
%% End of file `sample-sigconf.tex'.
